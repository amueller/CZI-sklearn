\documentclass{article}
\usepackage[subpreambles=true]{standalone}
\usepackage[utf8]{inputenc}
\usepackage[english]{babel}
\usepackage{import}
 
\title{Diversity and Inclusion Statement}

 
\begin{document}
 
    \maketitle
 

There is strong evidence that a community of diverse contributors and users producse more innovation. Unfortunately, like many other open source packages, the scikit-learn project lacks of diverse contributors. In particular, only one core-developer is a female, and most core developers are European.

To improve diversity within the project, Scikit-learn has been collaborating with the Women in Machine Learning and Data Science\footnote{\href{http://wimlds.org/opensourcesprints-2/}{http://wimlds.org/opensourcesprints-2/}} group (WiMLDS) over the past years. We have organized sprints in 2017, 2018 and 2019 in New York City and Nairobi (Kenya). Two new WiMLDS sprints are planned this year in NYC and San Francisco.

These sprints aim at engaging more people from under-represented groups and
encourage them to participate in the development of scikit-learn. During sprints, attendees learn about how to contribute to open source projects in general, and scikit-learn in particular. Attendees directly interact with the core developers and contribute improvements. While contributions during the sprints are usually small, they serve to integrate new developers into the community and create a pathway for future contributions.

\end{document}
